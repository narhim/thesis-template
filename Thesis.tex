
%!TEX root = Thesis.tex
%% ----------------------------------------------------------------
%% Author: Alex Schlosser
%%         schlosser@st.cs.uni-saarland.de
%% ---------------------------------------------------------------- 

% Set up the document
\documentclass[a4paper, 11pt, twoside]{Thesis}  % Use the "Thesis" style defined in Thesis.cls
\graphicspath{{./Figures/}}  % Location of the graphics files (set up for graphics to be in PDF format)
\setchapterpath{./Chapters/}
\setappendixpath{./Appendices/}

% Include any extra LaTeX packages required
\usepackage[square, numbers, comma, sort&compress]{natbib}  % Use the "Natbib" style for the references in the Bibliography
\usepackage{verbatim}  % Needed for the "comment" environment to make LaTeX comments
\usepackage{verbatimbox}
\usepackage[usenames]{xcolor}
\usepackage{tabularx}
\usepackage{multirow}
\usepackage{wrapfig}
\usepackage{framed}
\usepackage{amssymb}
\usepackage{amsmath}
\usepackage{amsthm}
\usepackage{stmaryrd}
\usepackage{framed}
\usepackage{hyperref}
\usepackage{todonotes}
\reversemarginpar
\presetkeys{todonotes}{inline, size=\tiny, prepend, caption={TODO}}{}
\usepackage{listings}
\lstset{language=C,numbers=left}
\usepackage{longtable}
%My packages

\usepackage{multicol}

\usepackage[shortlabels]{enumitem}
\usepackage{leipzig}
\newleipzig{ger}{ger}{gerund}
\newleipzig{impers}{impers}{impersonal}
\usepackage{gb4e}

%% Specify thesis parameters here
%% ----------------------------------------------------------------
\setboolean  {proposal}{false} % disables all stuff that is not needed for a proposal if set to true


\university  {Saarland University}{http://www.uni-saarland.de/}
\department  {Department of Language Science and Technology}{https://www.uni-saarland.de/en/department/lst.html}
\group       {Chair}{http://www.st.cs.uni-saarland.de/}
\faculty     {Faculty of Philosophy}{https://www.uni-saarland.de/fakultaet/p.html}
\supervisor  {Prof. Vera Demberg}
\advisor     {Dr. Lucia Donatelli}
\degree      {Master of Language Science and Technology}
\thesiskind  {Master Thesis}
\authors     {Teresa Rosa Martín Soeder}
\thesistitle {Crowdsourcing veridicality annotations in Spanish: How much do speakers disagree?}
\date        {\today}
%\subject     {}
%\keywords    {}
%% ----------------------------------------------------------------

% If this is a proposal prepend 'Proposal for a' to thesiskind
\doifproposal{
  \let\thesisnameproposal\thesisname
  \thesiskind{Proposal for a \thesisnameproposal}
}

% Enable git revision parsing (Note: you need to enable shell-excape and pipe support for this feature to work)
%\revision{\input{|"git rev-parse HEAD | cut -c 1-8"}}

\begin{document}
\frontmatter    % Begin Roman style (i, ii, iii, iv...) page numbering

%% Title Page
%% ----------------------------------------------------------------
\maketitle

%% Declaration Page required for the Thesis, your institution may give you a different text to place here
%% ----------------------------------------------------------------
\doifnotproposal{
\begin{declaration}
  \begin{center}
    \bf \large
    Eidesstattliche Erkl\"{a}rung
  \end{center}
  Ich erkl\"{a}re hiermit an Eides Statt,
  dass ich die vorliegende Arbeit selbstst\"{a}ndig verfasst und keine
  anderen als die angegebenen Quellen und Hilfsmittel verwendet habe.

  \begin{center}
    \bf \large
    Statement in Lieu of an Oath
  \end{center}
  I hereby confirm that I have written this thesis on my own and 
  that I have not used any other media or materials than the ones
  referred to in this thesis.

  \vfill
  \begin{center}
    \bf \large
    Einverst\"{a}ndniserkl\"{a}rung
  \end{center}
  Ich bin damit einverstanden, dass meine (bestandene) Arbeit in beiden 
  Versionen in die Bibliothek der Informatik aufgenommen und damit 
  ver\"{o}ffentlicht wird. 

  \begin{center}
    \bf \large
    Declaration of Consent
  \end{center}
  I agree to make both versions of my thesis (with a passing grade) 
  accessible to the public by having them added to the library of the
  Computer Science Department.

  \vfill
  \vfill
   
  Datum/Date:\\
  \rule[1em]{25em}{0.5pt}  % This prints a line to write the date

  Unterschrift/Signature:\\
  \rule[1em]{25em}{0.5pt}  % This prints a line for the signature
   
\end{declaration}
  \cleardoublepage  % Declaration ended, now start a new page
}

%% The Abstract Page
%% ----------------------------------------------------------------
\begin{abstract}
  \input{\chapterpath Abstract}
\end{abstract}

\doifnotproposal{

%% The Acknowledgements page, for thanking everyone
%% ----------------------------------------------------------------
\begin{acknowledgements}
\begin{center}
The list of who to thank is long, can words be enough? The obvious start is with both my supervisors, Lucia and Vera, thanks to their guidance \textit{this} has come through. Then there is of course Toloka since without their grant I would not have been able to finance my study. Last, but not least, all the people whose support and advice have brought me here:\\
My father, whose path into academia I try to follow;\\
My mother, whose love for languages I inherit;\\
Each of my seven siblings, by blood or by law: three countries, two continents, one heart;\\
My friends back south, who never deterred me from trying new things;\\
And the friends from different corners of the world whom I have been lucky enough to encounter here.\\
The path has been long and not always easy, you have not left my side and I will not forget it. May this be one of the many journeys we will share together, may each of your dreams become real in the most splendid way, and may God hold you in the palm of His hand.\\
\bigskip
\textit{THANK YOU! ¡MUCHÍSIMAS GRACIAS! VIELEN DANK!}\\
\end{center}
\end{acknowledgements}
\cleardoublepage
}

%% Table of Contents
%% ----------------------------------------------------------------

\tableofcontents
\clearpage

%% The Dedication page
%% ----------------------------------------------------------------
\doifnotproposal{
  \thispagestyle{empty}  % Page style needs to be empty for this page
  \dedicatory{To those struggling to find their place in the world, keep fighting, it's worth the effort.\\
  To the researchers forgotten by the institutions, your voice is heard and you are not alone.}
}

\mainmatter   % Begin normal, numeric (1,2,3...) page numbering

%% Chapters
%% ----------------------------------------------------------------
% Include the chapters of the thesis, as separate files (careful of case-sensitivity and spaces in filenames)
% Capitalized chaptertitles generally look better

\loadchapter{introduction}{Introduction}
\loadchapter{background}{Background}
\loadchapter{pilotstudy}{Pilot Study}
\loadchapter{mainstudy}{Main Study}
\loadchapter{conclusions}{Conclusions}

\clearpage

%% Appendices
%% ----------------------------------------------------------------
\appendix

%\section{Pilot Study}
\subsection{Inter-Annotator Agreement Scores}
%!TEX root = ../Thesis.tex

\begin{minipage}{\linewidth}\begin{tabular}{|c|c|}
\hline
\multicolumn{2}{|c|}{Agreement Scores}\\\hline
Name & Value \\\hline
Fleiss' $\kappa$ & $0.160$\\\hline
Fleiss' $\kappa$ with linear weights & $0.177$\\\hline
Conger's $\kappa$ with linear weights & $0.176$\\\hline
Gwet's $AC_2$ with linear weights & $0.484$\\\hline
\end{tabular}
\end{minipage}
\newline
\newline
\newline
\begin{minipage}{\linewidth}
\begin{tabular}{|c|c|c|c|}
\hline
\multicolumn{4}{|c|}{Agreement Scores for Experimental Conditions}\\\hline
Subset &  $\kappa^{f}_{w}$ & $\kappa^{c}_{w}$ & $AC_2$\\\hline
ALL & $0.177$ & $0.179$ & $0.484$\\\hline
Negation & $0.094$ & $0.097$ & $0.388$\\\hline
Possibility & $0.278$ & $0.278$ & $0.592$\\\hline
Individual & $0.161$ & $0.162$ & $0.500$\\\hline
Collective & $0.190$ & $0.192$ & $0.470$\\\hline
\end{tabular}
\end{minipage}
\newline
\newline
\newline
\begin{minipage}{\linewidth}
\begin{tabular}{|c|c|c|c|}
\hline
\multicolumn{4}{|c|}{Agreement Scores for Different Combinations of 3 Raters}\\\hline
Combination &  $\kappa^{f}_{w}$ & $\kappa^{c}_{w}$ & $AC_2$\\\hline
ALL & $0,177$ & $0,179$ & $0,484$\\\hline
R1, R2 and R3 & $0.170$ & $0.172$ & $0.501$\\\hline
R1, R2 and R4 & $0.237$ & $0.239$ & $0.550$\\\hline
R1, R2 and R5 & $0.219$ & $0.223$ & $0.493$ \\\hline
R1, R3 and R4 & $0.181$ & $0.182$ & $0.527$ \\\hline
R1, R3 and R5 & $0.177$ & $0.180$ & $0.478$ \\\hline
R1, R4 and R5 & $0.242$ & $0.246$ & $0.527$ \\\hline
R2, R3 and R4 & $0.133$ & $0.134$ & $0.469$ \\\hline
R2, R3 and R5 & $0.099$ & $0.102$ & $0.396$ \\\hline
R2, R4 and R5 & $0.162$ & $0.165$ & $0.447$ \\\hline
R3, R4 and R5 & $0.147$ & $0.150$ & $0.450$\\\hline  
\end{tabular}
\end{minipage}
\newline
\newline
\newline
\begin{minipage}{\linewidth}
\begin{tabular}{|c|c|c|c|}
\hline
\multicolumn{4}{|c|}{Agreement Scores for Different Label Reductions}\\\hline
Reduction &  $\kappa^{f}_{w}$ & $\kappa^{c}_{w}$ & $AC_2$\\\hline
ALL & $0,177$ & $0,179$ & $0,484$\\\hline
No NaS & $0.190$ & $0.191$ & $0.469$ \\\hline
PS with PR & $0.186$ & $0.187$ & $0.572$\\\hline
PS with PR, no NaS & $0.204$ & $0.205$ & $0.537$\\\hline
PR with CT, PS with Uu & $0.153$ & $0.155$ & $0.635$\\\hline
PR with CT, PS with Uu, no NaS & $0.162$ & $0.164$ & $0.518$\\\hline
PS with Uu & $0.204$ & $0.205$ & $0.590$\\\hline
PS with Uu, no NaS & $0.224$ & $0.225$ & $0.550$\\\hline 
\end{tabular}
\end{minipage}

\subsection{Cumulative Link Mixed Model}
%!TEX root = ../Thesis.tex

%\afterpage{
\begin{minipage}{\linewidth}
%\captionof{table}{Random effects resulted from fixing a cumulative link mixed model the whole dataset with different combinations of mood conditions, mood categories, specificity conditions and specficity categories; and random intercepts for raters and premise-hypothesis pairs.}
\begin{tabular}{|c|c|c|c|c|c|c|c|}
\hline
\multicolumn{8}{|c|}{Model Scorers}\\\hline
link  &  Threshold & Nobs & logLik & AIC & niter & max.grad & cond.H\\\hline
logit & flexible & $1530$ & $-2456.68$ & $4961.36$ & $4502(13559)$ & $5.68\times 10^{-3}$ & $8.5\times 10^{2}$\\\hline
\end{tabular}
\label{tab:modrand}
\end{minipage}
\newline
\newline
\newline
\newline
\begin{minipage}{\linewidth}
%\captionof{table}{Estimations, standard errors, z and p values for the different cofficients of the cumulative link mixed model. "con" stands for condition, "cat" stands for category, and ":" indicates crossed effect.}
	\begin{tabular}{|c|c|c|c|c|}
	\hline
	\multicolumn{5}{|c|}{Model Coefficients}\\\hline
	Coefficient & Estimate & Std. Error & z value &Pr($>|z|$)\\\hline
	moodconadverb                                 &$2.151$ &$0.282$& $7.632$&$2.31\times10^{-14}$\\\hline
	moodcatindicative                             &$0.427$ &$ 0.311$ & $ 1.371$&$0.170$ \\\hline
	moodcatsubjunctive                            &$-0.063$ & $ 0.305$ & $-0.206$&$0.836$\\\hline
	specificityconcollective                      &$-0.199$ & $0.250$ & $-0.797$&$0.425$\\\hline
	specificitycatmixed                           &$0.280$& $0.329$ & $0.851$&$0.395$\\\hline
	specificitycatproper                          &$0.024$ &$0.328$&$0.072$&$0.9426$\\\hline
	moodconadverb:moodcatindicative               &$-3.028$ & $0.334$ &$-9.072$&$< 2\times10^{-16}$\\\hline
	moodconadverb:moodcatsubjunctive              &$-2.868$ &$0.330$& $-8.676$&$< 2\times10^{-16}$\\\hline
	specificityconcollective:specificitycatmixed  &$-0.291$& $0.311$ & $-0.933$&$0.351$\\\hline
	specificityconcollective:specificitycatproper &$0.107$ & $0.312$ &$0.344$&$0.731$\\\hline
	moodconadverb:specificityconcollective        &$0.494$ & $0.256$ & $1.928$&$0.054$\\\hline
	moodcatindicative:specificitycatmixed        & $-0.209$ & $0.390$ & $-0.536$&$0.592$\\\hline
	moodcatsubjunctive:specificitycatmixed       & $-0.366$ & $0.387$ & $-0.947$&$0.344$\\\hline
	moodcatindicative:specificitycatproper       & $-0.346$ & $0.391$ & $-0.884$&$0.377$\\\hline
	moodcatsubjunctive:specificitycatproper      & $-0.131$ & $0.387$ & $-0.338$&$0.735$\\\hline
	\end{tabular}
	\label{tab:mod2} 
\end{minipage}
\newpage
\begin{minipage}{\linewidth}
%\captionof{table}{Random effects resulted from fixing a cumulative link mixed model the whole dataset with different combinations of mood conditions, mood categories, specificity conditions and specficity categories; and random intercepts for raters and premise-hypothesis pairs.}
\begin{tabular}{|c|c|c|c|}
\hline
\multicolumn{4}{|c|}{Model Random Effects}\\\hline
Groups  &  Name       &  Variance & Std.Dev.\\\hline
PAIR  & Intercept & $0.512$   & $0.715$  \\\hline
RATER & Intercept & $0.505$   & $0.710$ \\\hline
\end{tabular}
\label{tab:modrand}
\end{minipage}
\newline
\newline
\newline
\begin{minipage}{\linewidth}
%\captionof{table}{Threshold coefficients resulted from fixing a cumulative link mixed model the whole dataset with different combinations of mood conditions, mood categories, specificity conditions and specficity categories; and random intercepts for raters and premise-hypothesis pairs.}
\begin{tabular}{|c|c|c|c|}
\hline
\multicolumn{4}{|c|}{Model Threshold Coefficients}\\\hline
Threshold &  Estimate & Std. Error & z value\\\hline
NaS|CT- & $-4.254$ & $0.326$ & $-13.062$\\\hline
CT-|PR- & $-3.272$ & $0.298$ & $-10.962$\\\hline
PR-|PS- & $-2.756$ & $0.291$ & $ -9.480$\\\hline
PS-|Uu  & $-2.371$ & $0.286$ & $ -8.277$\\\hline
Uu|PS+  & $-0.904$ & $0.278$ & $ -3.249$\\\hline
PS+|PR+ & $ 0.006$ & $0.277$ & $  0.023$\\\hline
PR+|CT+ & $ 1.233$ & $0.279$ & $  4.415$\\\hline
\end{tabular}
\label{tab:appmodthres}
\end{minipage}

\section{Main Study: CLMMs with Most Frequent Matrices}
\subsection{Predictors: Mood}
%\afterpage{
\begin{minipage}{\linewidth}
%\captionof{table}{Random effects resulted from fixing a cumulative link mixed model the whole dataset with different combinations of mood conditions, mood categories, specificity conditions and specficity categories; and random intercepts for raters and premise-hypothesis pairs.}
\begin{tabular}{|c|c|c|c|c|c|c|c|}
\hline
\multicolumn{8}{|c|}{Model Scorers}\\\hline
link  &  Threshold & Nobs & logLik & AIC & niter & max.grad & cond.H\\\hline
logit & flexible & $1701$ & $-3125.09$ & $6270.18$ & $1024(2105)$ & $5.04\times 10^{-3}$ & $1.4\times 10^{2}$\\\hline
\end{tabular}
%\label{tab:modrand}
\end{minipage}
\newline
\newline
\newline
\begin{minipage}{\linewidth}
%\captionof{table}{Estimations, standard errors, z and p values for the different cofficients of the cumulative link mixed model. "con" stands for condition, "cat" stands for category, and ":" indicates crossed effect.}
	\begin{tabular}{|c|c|c|c|c|}
	\hline
	\multicolumn{5}{|c|}{Model Coefficients}\\\hline
	Coefficient & Estimate & Std. Error & z value &Pr($>|z|$)\\\hline
	MOOD\_CONDITIONindicative  & $-0.5769$ & $0.1156$ & $-4.990$ & $6.05\times10^{07}$\\\hline
	MOOD\_CONDITIONsubjunctive & $-0.6338$ & $0.1161$ & $-5.462$ & $4.72\times10^{08}$\\\hline
	\end{tabular}
	%\label{tab:mod2} 
\end{minipage}
\newline
\newline
\newline
\begin{minipage}{\linewidth}
%\captionof{table}{Random effects resulted from fixing a cumulative link mixed model the whole dataset with different combinations of mood conditions, mood categories, specificity conditions and specficity categories; and random intercepts for raters and premise-hypothesis pairs.}
\begin{tabular}{|c|c|c|c|}
\hline
\multicolumn{4}{|c|}{Model Random Effects}\\\hline
Groups  &  Name       &  Variance & Std.Dev.\\\hline
PAIR  & Intercept & $0.080$   & $0.283$  \\\hline
RATER & Intercept & $0.014$   & $0.120$ \\\hline
\end{tabular}
%\label{tab:modrand}
\end{minipage}
\newline
\newline
\newline
\begin{minipage}{\linewidth}
%\captionof{table}{Threshold coefficients resulted from fixing a cumulative link mixed model the whole dataset with different combinations of mood conditions, mood categories, specificity conditions and specficity categories; and random intercepts for raters and premise-hypothesis pairs.}
\begin{tabular}{|c|c|c|c|}
\hline
\multicolumn{4}{|c|}{Model Threshold Coefficients}\\\hline
Threshold &  Estimate & Std. Error & z value\\\hline
CT-|PR- & $-3.075$ & $0.126$ &$-24.447$\\\hline
PR-|PS- & $-2.100$ & $0.102$ &$-19.769$\\\hline
PS-|Uu  & $-1.383$ & $0.094$ &$-14.741$\\\hline
Uu|PS+  & $-0.888$ & $0.090$ &$ -9.905$\\\hline
PS+|PR+ & $-0.200$ & $0.087$ &$ -2.308$\\\hline
PR+|CT+ & $ 0.560$ & $0.088$ &$  6.376$\\\hline
\end{tabular}
%\label{tab:appmodthres}
\end{minipage}

%!TEX root = ../Thesis.tex
\subsection{Predictors: Mood and Matrix}
%\afterpage{
\begin{minipage}{\linewidth}
%\captionof{table}{Random effects resulted from fixing a cumulative link mixed model the whole dataset with different combinations of mood conditions, mood categories, specificity conditions and specficity categories; and random intercepts for raters and premise-hypothesis pairs.}
\begin{tabular}{|c|c|c|c|c|c|c|c|}
\hline
\multicolumn{8}{|c|}{Model Scorers}\\\hline
link  &  Threshold & Nobs & logLik & AIC & niter & max.grad & cond.H\\\hline
logit & flexible & $1701$ & $-3112.90$ & $6269.81$ & $3528(7150)$ & $5.04\times 3.19^{-3}$ & $2.6\times 10^{3}$\\\hline
\end{tabular}
%\label{tab:modrand}
\end{minipage}
\newline
\newline
\newline
\begin{minipage}{\linewidth}
%\captionof{table}{Estimations, standard errors, z and p values for the different cofficients of the cumulative link mixed model. "con" stands for condition, "cat" stands for category, and ":" indicates crossed effect.}
	\begin{tabular}{|c|c|c|c|}
	\hline
	\multicolumn{4}{|c|}{Model Coefficients}\\\hline
	Coefficient & Estimate & Std. Error & Pr($>|z|$)\\\hline
	MOOD\_CONDITIONindicative                  & $-0.699$ & $0.297$ & $  0.019$\\\hline
	MOOD\_CONDITIONsubjunctive                 & $-1.309$ & $0.305$ & $1.76\times 10^{05}$\\\hline
	MATRIXconsiderar                          & $-0.404$ & $0.292$ & $  0.167$\\\hline
	MATRIXcreer                               & $-0.584$ & $0.280$ & $  0.037$\\\hline
	MATRIXdecir                               & $-0.139$ & $0.312$ & $  0.656$\\\hline
	MATRIXsaber                               & $-0.053$ & $0.252$ & $  0.834$\\\hline
	MOOD\_CONDITIONindicative:MATRIXconsiderar & $ 0.252$ & $0.405$ & $  0.534$\\\hline
	MOOD\_CONDITIONsubjunctive:MATRIXconsiderar& $ 0.807$ & $0.401$ & $  0.044$\\\hline
	MOOD\_CONDITIONindicative:MATRIXcreer      & $ 0.154$ & $0.390$ & $  0.693$\\\hline
	MOOD\_CONDITIONsubjunctive:MATRIXcreer     & $ 0.836$ & $0.393$ & $  0.033$\\\hline
	MOOD\_CONDITIONindicative:MATRIXdecir      & $-0.125$ & $0.422$ & $  0.768$\\\hline
	MOOD\_CONDITIONsubjunctive:MATRIXdecir     & $ 0.677$ & $0.441$ & $  0.125$\\\hline
	MOOD\_CONDITIONindicative:MATRIXsaber      & $ 0.156$ & $0.345$ & $  0.652$\\\hline
	MOOD\_CONDITIONsubjunctive:MATRIXsaber     & $ 0.767$ & $0.352$ & $  0.029$\\\hline
	\end{tabular}
	%\label{tab:mod2} 
\end{minipage}
\newline
\newline
\newline
\begin{minipage}{\linewidth}
%\captionof{table}{Random effects resulted from fixing a cumulative link mixed model the whole dataset with different combinations of mood conditions, mood categories, specificity conditions and specficity categories; and random intercepts for raters and premise-hypothesis pairs.}
\begin{tabular}{|c|c|c|c|}
\hline
\multicolumn{4}{|c|}{Model Random Effects}\\\hline
Groups  &  Name       &  Variance & Std.Dev.\\\hline
PAIR  & Intercept & $0.028$   & $0.167$  \\\hline
RATER & Intercept & $0.014$   & $0.120$ \\\hline
\end{tabular}
%\label{tab:modrand}
\end{minipage}
\newline
\newline
\newline
\begin{minipage}{\linewidth}
%\captionof{table}{Threshold coefficients resulted from fixing a cumulative link mixed model the whole dataset with different combinations of mood conditions, mood categories, specificity conditions and specficity categories; and random intercepts for raters and premise-hypothesis pairs.}
\begin{tabular}{|c|c|c|c|}
\hline
\multicolumn{4}{|c|}{Model Threshold Coefficients}\\\hline
Threshold &  Estimate & Std. Error & z value\\\hline
CT-|PR- & $-3.289$ & $0.239$ & $-13.781$\\\hline
PR-|PS- & $-2.225$ & $0.227$ & $ -9.804$\\\hline
PS-|Uu  & $-1.600$ & $0.223$ & $ -7.157$\\\hline
Uu|PS+  & $-1.105$ & $0.222$ & $ -4.987$\\\hline
PS+|PR+ & $-0.417$ & $0.220$ & $ -1.899$\\\hline
PR+|CT+ & $ 0.343$ & $0.219$ & $  1.564$\\\hline
\end{tabular}
%\label{tab:appmodthres}
\end{minipage}
 % Appendix Title

%%!TEX root = ../Thesis.tex

\begin{minipage}{\linewidth}\begin{tabular}{|c|c|}
\hline
\multicolumn{2}{|c|}{Agreement Scores}\\\hline
Name & Value \\\hline
Fleiss' $\kappa$ & $0.160$\\\hline
Fleiss' $\kappa$ with linear weights & $0.177$\\\hline
Conger's $\kappa$ with linear weights & $0.176$\\\hline
Gwet's $AC_2$ with linear weights & $0.484$\\\hline
\end{tabular}
\end{minipage}
\newline
\newline
\newline
\begin{minipage}{\linewidth}
\begin{tabular}{|c|c|c|c|}
\hline
\multicolumn{4}{|c|}{Agreement Scores for Experimental Conditions}\\\hline
Subset &  $\kappa^{f}_{w}$ & $\kappa^{c}_{w}$ & $AC_2$\\\hline
ALL & $0.177$ & $0.179$ & $0.484$\\\hline
Negation & $0.094$ & $0.097$ & $0.388$\\\hline
Possibility & $0.278$ & $0.278$ & $0.592$\\\hline
Individual & $0.161$ & $0.162$ & $0.500$\\\hline
Collective & $0.190$ & $0.192$ & $0.470$\\\hline
\end{tabular}
\end{minipage}
\newline
\newline
\newline
\begin{minipage}{\linewidth}
\begin{tabular}{|c|c|c|c|}
\hline
\multicolumn{4}{|c|}{Agreement Scores for Different Combinations of 3 Raters}\\\hline
Combination &  $\kappa^{f}_{w}$ & $\kappa^{c}_{w}$ & $AC_2$\\\hline
ALL & $0,177$ & $0,179$ & $0,484$\\\hline
R1, R2 and R3 & $0.170$ & $0.172$ & $0.501$\\\hline
R1, R2 and R4 & $0.237$ & $0.239$ & $0.550$\\\hline
R1, R2 and R5 & $0.219$ & $0.223$ & $0.493$ \\\hline
R1, R3 and R4 & $0.181$ & $0.182$ & $0.527$ \\\hline
R1, R3 and R5 & $0.177$ & $0.180$ & $0.478$ \\\hline
R1, R4 and R5 & $0.242$ & $0.246$ & $0.527$ \\\hline
R2, R3 and R4 & $0.133$ & $0.134$ & $0.469$ \\\hline
R2, R3 and R5 & $0.099$ & $0.102$ & $0.396$ \\\hline
R2, R4 and R5 & $0.162$ & $0.165$ & $0.447$ \\\hline
R3, R4 and R5 & $0.147$ & $0.150$ & $0.450$\\\hline  
\end{tabular}
\end{minipage}
\newline
\newline
\newline
\begin{minipage}{\linewidth}
\begin{tabular}{|c|c|c|c|}
\hline
\multicolumn{4}{|c|}{Agreement Scores for Different Label Reductions}\\\hline
Reduction &  $\kappa^{f}_{w}$ & $\kappa^{c}_{w}$ & $AC_2$\\\hline
ALL & $0,177$ & $0,179$ & $0,484$\\\hline
No NaS & $0.190$ & $0.191$ & $0.469$ \\\hline
PS with PR & $0.186$ & $0.187$ & $0.572$\\\hline
PS with PR, no NaS & $0.204$ & $0.205$ & $0.537$\\\hline
PR with CT, PS with Uu & $0.153$ & $0.155$ & $0.635$\\\hline
PR with CT, PS with Uu, no NaS & $0.162$ & $0.164$ & $0.518$\\\hline
PS with Uu & $0.204$ & $0.205$ & $0.590$\\\hline
PS with Uu, no NaS & $0.224$ & $0.225$ & $0.550$\\\hline 
\end{tabular}
\end{minipage} % Appendix Title
%%!TEX root = ../Thesis.tex
%\afterpage{
\begin{minipage}{\linewidth}
%\captionof{table}{Random effects resulted from fixing a cumulative link mixed model the whole dataset with different combinations of mood conditions, mood categories, specificity conditions and specficity categories; and random intercepts for raters and premise-hypothesis pairs.}
\begin{tabular}{|c|c|c|c|c|c|c|c|}
\hline
\multicolumn{8}{|c|}{Model Scorers}\\\hline
link  &  Threshold & Nobs & logLik & AIC & niter & max.grad & cond.H\\\hline
logit & flexible & $1530$ & $-2456.68$ & $4961.36$ & $4502(13559)$ & $5.68\times 10^{-3}$ & $8.5\times 10^{2}$\\\hline
\end{tabular}
\label{tab:modrand}
\end{minipage}
\newline
\newline
\newline
\newline
\begin{minipage}{\linewidth}
%\captionof{table}{Estimations, standard errors, z and p values for the different cofficients of the cumulative link mixed model. "con" stands for condition, "cat" stands for category, and ":" indicates crossed effect.}
	\begin{tabular}{|c|c|c|c|c|}
	\hline
	\multicolumn{5}{|c|}{Model Coefficients}\\\hline
	Coefficient & Estimate & Std. Error & z value &Pr($>|z|$)\\\hline
	moodconadverb                                 &$2.151$ &$0.282$& $7.632$&$2.31\times10^{-14}$\\\hline
	moodcatindicative                             &$0.427$ &$ 0.311$ & $ 1.371$&$0.170$ \\\hline
	moodcatsubjunctive                            &$-0.063$ & $ 0.305$ & $-0.206$&$0.836$\\\hline
	specificityconcollective                      &$-0.199$ & $0.250$ & $-0.797$&$0.425$\\\hline
	specificitycatmixed                           &$0.280$& $0.329$ & $0.851$&$0.395$\\\hline
	specificitycatproper                          &$0.024$ &$0.328$&$0.072$&$0.9426$\\\hline
	moodconadverb:moodcatindicative               &$-3.028$ & $0.334$ &$-9.072$&$< 2\times10^{-16}$\\\hline
	moodconadverb:moodcatsubjunctive              &$-2.868$ &$0.330$& $-8.676$&$< 2\times10^{-16}$\\\hline
	specificityconcollective:specificitycatmixed  &$-0.291$& $0.311$ & $-0.933$&$0.351$\\\hline
	specificityconcollective:specificitycatproper &$0.107$ & $0.312$ &$0.344$&$0.731$\\\hline
	moodconadverb:specificityconcollective        &$0.494$ & $0.256$ & $1.928$&$0.054$\\\hline
	moodcatindicative:specificitycatmixed        & $-0.209$ & $0.390$ & $-0.536$&$0.592$\\\hline
	moodcatsubjunctive:specificitycatmixed       & $-0.366$ & $0.387$ & $-0.947$&$0.344$\\\hline
	moodcatindicative:specificitycatproper       & $-0.346$ & $0.391$ & $-0.884$&$0.377$\\\hline
	moodcatsubjunctive:specificitycatproper      & $-0.131$ & $0.387$ & $-0.338$&$0.735$\\\hline
	\end{tabular}
	\label{tab:mod2} 
\end{minipage}
\newpage
\begin{minipage}{\linewidth}
%\captionof{table}{Random effects resulted from fixing a cumulative link mixed model the whole dataset with different combinations of mood conditions, mood categories, specificity conditions and specficity categories; and random intercepts for raters and premise-hypothesis pairs.}
\begin{tabular}{|c|c|c|c|}
\hline
\multicolumn{4}{|c|}{Model Random Effects}\\\hline
Groups  &  Name       &  Variance & Std.Dev.\\\hline
PAIR  & Intercept & $0.512$   & $0.715$  \\\hline
RATER & Intercept & $0.505$   & $0.710$ \\\hline
\end{tabular}
\label{tab:modrand}
\end{minipage}
\newline
\newline
\newline
\begin{minipage}{\linewidth}
%\captionof{table}{Threshold coefficients resulted from fixing a cumulative link mixed model the whole dataset with different combinations of mood conditions, mood categories, specificity conditions and specficity categories; and random intercepts for raters and premise-hypothesis pairs.}
\begin{tabular}{|c|c|c|c|}
\hline
\multicolumn{4}{|c|}{Model Threshold Coefficients}\\\hline
Threshold &  Estimate & Std. Error & z value\\\hline
NaS|CT- & $-4.254$ & $0.326$ & $-13.062$\\\hline
CT-|PR- & $-3.272$ & $0.298$ & $-10.962$\\\hline
PR-|PS- & $-2.756$ & $0.291$ & $ -9.480$\\\hline
PS-|Uu  & $-2.371$ & $0.286$ & $ -8.277$\\\hline
Uu|PS+  & $-0.904$ & $0.278$ & $ -3.249$\\\hline
PS+|PR+ & $ 0.006$ & $0.277$ & $  0.023$\\\hline
PR+|CT+ & $ 1.233$ & $0.279$ & $  4.415$\\\hline
\end{tabular}
\label{tab:appmodthres}
\end{minipage}
 % Appendix Title


\backmatter

%% Lists of figures and tables
%% ----------------------------------------------------------------
\doifnotproposal{
  \lhead{\emph{List of Figures}}
  \addtotoc{List of Figures}
  \listoffigures
  \clearpage

    \addtotoc{List of Tables}
    \lhead{\emph{List of Tables}}
  \listoftables
  \clearpage
}

%% Bibliography
%% ----------------------------------------------------------------
\label{Bibliography}
\addtotoc{Bibliography}
\lhead{\emph{Bibliography}}  % Change the left side page header to "Bibliography"
\bibliographystyle{unsrtnat}  % Use the "unsrtnat" BibTeX style for formatting the Bibliography
\bibliography{Bibliography}  % The references (bibliography) information are stored in the file named "Bibliography.bib"

\beginappendix %Begins the appendix
%\loadappendix{AppendixA}{This is a trial}
%\loadappendix{AppendixB}{This is another trial}
\loadappendix{pilotiaa}{Pilot Study: Inter-Annotator Agreement Scores}\label{app:piliaa}
\loadappendix{pilotmodel}{Pilot Study: Cumulative Link Mixed Model}\label{app:pilmodel}
\loadappendix{extramodels}{Main Study: CLMMs with Most Frequent Matrices}


\end{document}  % The End