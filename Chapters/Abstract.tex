In veridicality studies, an area of research of Natural Language Inference (NLI), the factuality of different contexts is evaluated. Here, we aim to reduce the lack of research in this field, particularly in Spanish, by presenting the analysis of annotations collected in a pilot study for two different contexts, mood alternation and specificity. Results show a inter-annotator agreement score of $AC_2=0.484$, slightly lower than that of \citet{de2012did} ($\kappa = 0.53$), a main reference to this work. Furthermore, a significant effect of some mood alternation conditions, and a few unexpected tendencies, like high factuallity despite the presence of a negation marker, were found. Finally, suggestions that could explain the lack of inter-annotator agreement as well as improve it for the final study are defined. The annotations collected are available on \url{https://github.com/narhim/veridicality_spanish}.