In veridicality studies, an area of research of Natural Language Inference (NLI), the factuality of different contexts is evaluated. This task, known to be a difficult one since often it is not clear what the interpretation should be \citet{uma2021learning}, is key for building any Natural Language Understanding (NLU) system that aims at making the right inferences. Here the results of two studies that analyze the veridicality of mood alternation and specificity in Spanish, and whose labels are based on those of \citet{sauri2009factbank} are presented. The first one, the pilot study, has an inter-annotator agreement score of $AC_2=0.484$, slightly lower than that of \citet{de2012did} ($\kappa = 0.53$), a main reference to this work; shows some mood-related significant effects, and presents a few unexpected tendencies, like  high factuality despite the presence of a negation marker. The second one, the main study, has an inter-annotator agreement of $AC_2=0.114$, quite lower than the first study, and a couple of mood-related significant effects. Due to this strong lack of agreement, an analysis of what factors cause disagreement is presented together with a discussion based on the work of \citet{de2012did} and \citet{pavlick2019inherent} about the quality of the annotations gathered and whether other types of analysis like entropy distribution could better represent this corpus. The annotations collected for both studies are available at \url{https://github.com/narhim/veridicality_spanish}.