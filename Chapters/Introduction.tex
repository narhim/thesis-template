%!TEX root = ../Thesis.tex
\label{chap:intr}
\section{Introduction}
\textit{Is that true? Did it really happen?} Often we ask ourselves or our interlocutors these questions, we try to assess if the information conveyed is likely to be truthful or not, that is, if it corresponds to actual situations in the real world \citep{sauri2009factbank}. Also, as speakers or authors we usually try to portray what we know about the \textbf{truthfulness} or \textbf{factuality} of the events conveyed. Therefore this kind of language understanding needs to be incorporated into any system that aims at understanding human language.\\

Let us consider examples (\ref{ex:quest}) to (\ref{ex:ans}), extracted from the dataset SQuAD \citep{rajpurkar2016squad}, where an interaction for a question answering system is portrayed. In order to give such a simple answer as "seven" the systems is making an inference from the context given. Specificially it is inferring that "Seven Monument Zones are present in the Kathmandu Valley" follows or \textit{loosely} entails from "These are [...] in the seven well-defined Monument Zones of the Kathmandu valley". The term \textbf{follows} and \textbf{loosely entails} are used to denote that we are referring to pragmatic inferences and not logical entailment \citep{manning2006local}.\\ 

\begin{exe}
        \ex\label{ex:quest}
        \begin{xlist}
                \item Question: "How many Monument Zones are present in the Kathmandu valley?" \label{ex:quest}
                \item Context: These are amply reflected in the many temples, shrines, stupas, monasteries, and palaces in the seven well-defined Monument Zones of the Kathmandu valley are part of a UNESCO World Heritage Site.\label{ex:cont}
                \item Answer: "seven"\label{ex:ans}
        \end{xlist}
\end{exe}

But, how does the question answering or we as speakers understand that the inference is correct, or in other words, that it is true that there are sevent monument zones in the Kathmandu Valley? There are different linguistic factors at play, like the use of negation or modality markers exemplified both in Spanish and in English in utterances (\ref{ex:colpos}) to (\ref{ex:colmod}). All three of them have one event in common "Pedro has done the laundry" (\textit{Pedro ha hecho la colada}), but in example (\ref{ex:colneg}) we have the negation adverb "not" (\textit{no}) and in example (\ref{ex:colmod}) we have the modal verb "must" (\textit{tener que}). The question now is wether the event "Pedro has done the laundry" can be inferred from each of these utterances.\\

On the first case, the answer is a clear yes, unless there is something in our \textbf{world knowledge} about Pedro that lead us to not be certain about it or to complete negate its veracity. In example (\ref{ex:colneg}) the situation is the exact opposite. Due to the presence of the negation adverb "not" (\textit{no}) the most likely situation is that we negate the truthfulness of "Pedro has done the laundry", unless, yet again, we know something about Pedro that changes this. As to utterance (\ref{ex:colmod}), the situation is roughly in the middle between the two previous sentences. The presence of the deontic modal "must" (\textit{tener que}) modifies the truthfulness or factuality of the event. It does not make it completely true, but it does not make it completely false. It instead yields the "Pedro has done the laundry" (\textit{Pedro ha hecho la colada}) an obligation \citep{morante2012modality}, and thus makes its factuality uncertain and more dependent upon the listener than the two previous utterances, both for Spanish and English speakers.\\

\begin{exe}
  \ex
    \begin{xlist}
      \item  {\gll Pedro ha hecho la colada.\\ Pedro have.\Prs.\Ind.\Tsg{} do.\Ptcp{} the.\F.\Sg{} laundry.\M.\Sg{}\\\ "Pedro has done the laundry."\glt }\label{ex:colpos}
      \item  {\gll Pedro no ha hecho la colada.\\ Pedro not have.\Prs.\Ind.\Tsg{} do.\Ptcp{} the.\F.\Sg{} laundry.\M.\Sg{}\\\ "Pedro hasn't done the laundry."\glt }\label{ex:colneg}
      \item  {\gll Pedro tiene que haber hecho la colada.\\ Pedro tiene.\Prs.\Ind.\Tsg{} that have.\Inf{} do.\Ptcp{} the.\F.\Sg{} laundry.\M.\Sg{}\\\ "Pedro must have done done the laundry."\glt }\label{ex:colmod}
    \end{xlist}
\end{exe}

A study of what linguistic factors affect the factuality of an event, like the analysis of "not" and "must" that has just been presented, it is what is called veridicality, research topic for this thesis. To be more specific, veridicality is an area of research within natural language inference (NLI) and theoretical linguistics that studies the truth value of a proposition or event in a specific context \citep{giannakidou2014non,giannakidou2015mixed}. As to NLI, it is a branch of natural language understanding (NLU) with its main task being entailment classification, that is, as it has been done above, given premises like utterances (\ref{ex:colpos}) to (\ref{ex:colmod}) in Spanish or English, and the hypothesis "Pedro has done the laundry" (\textit{Pedro ha hecho la colada}), the task is to classify the relationship between each premise with the hypothesis by picking a label from a usually small set of labels like \textit{\{entailment, neutral, contradiction\}} \citep{williams2017broad} or \textit{\{yes, unknown, not\}}, depending on how the task is defined. In our case, the most likely scenario is that we pick the labels yes, not, unknown respectively. For a system to successfully complete this task, as it has been hinted above with example (\ref{ex:quest}), it needs to form a thorough and complete meanig representation of both sentences \citep{williams2017broad}, and here is where veridicality, among other disciplines, comes into play.\\ 

The focus of thesis is the analysis of veridicality judgments in Spanish; that is, the analysis of factuality judgments in specific contexts in Spanish. In particularly, the goal is to analyze how mood alternation, in other words, the possibility of using a verb either in indicative or subjunctive mood; and the specificity of the syntactic subject, that is, the identifiability of the referent in the discourse universe \citep{caudet1999expresiones}, affect factuality judgments about an event. These factors are presented in utterances (\ref{ex:cond1}) and (\ref{ex:cond2}).\\

Assuming for the moment that the alternative subjects in examples (\ref{ex:cond1}) and (\ref{ex:cond2}) are one entity, we see that these two sentences are almost identical. The only difference lays on the embedded verb \textit{tener} (to have). As indicated by the glosses, on the first case the verb is in the indicative mood and on the second one the verb is on the subjunctive mood. Whereas normally a verb is only allowed to be either in the indicative or in the subjunctive mood, here both possibilities are allowed and this is what is called mood alternation. More details about this phenomen will be given in the Chapter \ref{chap:back}, but it should be noted now that there is no straight translation of this difference into English, thus translations are kept exactly the same as in \citet{faulkner2021systematic}.\\

\begin{exe}
  \ex
    \begin{xlist}
      \item{\gll El gobierno/ El presidente no dijo que el país tenía problemas económicos.\\ the.\M.\Sg{} government/ the.\M.\Sg{} president.\M.\Sg{} not say.\Pst.\Pfv.\Ind.\Tsg{} that the.\M.\Sg{} country.\F.\Sg{} have.\Pst.\Ipfv.\Ind.\Tsg{} problem.\M.\Pl{} economic.\M.\Pl{} \\\glt The government/ The president didn't say that the country had economic problems.}\label{ex:cond1}
      \item{\gll El gobierno/ El presidente no dijo que el país tuviera problemas económicos.\\ the.\M.\Sg{} government/ the.\M.\Sg{} president.\M.\Sg{} not say.\Pst.\Pfv.\Ind.\Tsg{} that the.\M.\Sg{} country.\F.\Sg{} have.\Pst.\Ipfv.\Sbjv.\Tsg{} problem.\M.\Pl{} economic.\M.\Pl{} \\\glt The government/ The president didn't say that the country had economic problems.}\label{ex:cond2}
    \end{xlist}
\end{exe}

As to the alternative subjects \textit{el gobierno / el presidente} (the government / the president), they reflect differences on specificity. The first option, \textit{el gobierno} (the government), refers to a set of people who rule or administer a country, that is, it refers to a set of entities. On opposition to this, there is the second option, \textit{el presidente} (the president), which refers to a single entity. As with mood alternation, more details about this phenomenon and how it affects the factuality of an event are provided in the next chapter.\\

The above-mentioned goal of understanding how mood alternation and specificity affect the factuality of an event the following research questions are defined:\\  

\begin{enumerate}[RQ1.-]
        \item In a complex sentence, how does the mood alternation of the embedded verb that occurs due to the negation of the main or matrix verb affect the factuality value of the embedded event?
        \item In a simple sentence, how does the mood alternation caused by an adverb of doubt or possibility affect the factuality judgment of the event?\label{item:rq2}
        \item How does an individual subject affect the factuality judgment of the event?
        \item How does a subject that refers to a collective entity like an institution, affect the factuality judgment of the event?\label{item:rq4}
\end{enumerate}

In order to answer these questions and verify their relevance, to also verify experimental soundness and be able to get direct feedback from the annotators, a pilot study was run among friends and relatives who are linguistically naive native Spanish speakers. Given the results obtained which are introduced in Chapter \ref{chap:pil}, it was concluded that answering these four questions required a complicated experimental design and analysis. Thus it was decided to remove research question \ref{item:rq2} and only considered the following three:\\

\begin{enumerate}[RQ1.-]
        \item In a complex sentence, how does the mood alternation of the embedded verb that occurs due to the negation of the main or matrix verb affect the factuality value of the embedded event?
        \item How does an individual subject affect the factuality judgment of the event?
        \item How does a subject that refers to a collective entity like an institution, affect the factuality judgment of the event?\label{item:rq4}
\end{enumerate}

For the main study a similar methodology was followed: annotations were gathered from linguistically naive native Spanish speakers. But instead of acquaintances, annotators were recruited from the Toloka platform \citep{Pavlichenko2021crowdspeech}. Consequently, the sociolinguistical background increased significantly from mostly speakers from Spain to speakers from all countries where Spanish is either an official language or an important minority language.\\

Regarding the motivations behind this thesis, firstly Spanish was chosen not just for being my mother tongue, but because, to my knowledge, there is a considerable lack of veridicality studies and available corpora. As it will be explained in the next chapter, most the studies found were done from a different perspective since annotations informed not about the relation between a premise and a hypothesis, but directly annotated lexical elements with factuality labels. Even more, most of these corpora are annotated by linguistic experts. Here instead, the aim was to gather annotations from linguistically naive speakers in order to obtain pragmatically informed labels as it is often the case in natural language processing (NLP). More details about the differences between these perspectives will be given in the next chapter.\\

The most practical contribution from this thesis is two annotated corpora available at \url{https://github.com/narhim/veridicality_spanish}. For both of them the raw annotations are presented, that is, instead of the aggregated label for each pair, the label assigned by each annotator is offered. In adddition, every supplementary information, like translations or morphological annotations, that has been generated is given. The first dataset, resulting from the pilot study, consists on $306$ pairs annotated by $5$ annotators each. The second one, resulting from the main study, consist on $477$ pairs annotated by $7$ annotators each.\\

Another important contribution is the analysis presented. It is done from a statistical perspective and a linguistic one, with the idea of not \textit{just} answering the research questions, but trying to understand the data as much as possible within the limitations of this thesis. This gives us a more complete picture of the annotations gathered and might help future researchers when trying to decide how to understand their data.\\ 

Next, Chapter \ref{chap:back} introduces the literature review and the explanation of the main concepts used here. Then, Chapter \ref{chap:pil} presents the already mentioned pilot study and then the main study is introduced in Chapter \ref{chap:main}. Finally, Chapter \ref{chap:con} brings together all the issues discussed in each experiment, suggests possible improvements, and proposes some lines of future work.\\
