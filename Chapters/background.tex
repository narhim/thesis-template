%!TEX root = ../Thesis.tex
\label{chap:back}
The goal of this chapter is to clarify the most important terms that will be used in this thesis, give theoretical support for the claims made for the experiments, and present some previous research important to the study here. Especially the focus is on distinguishing concepts that are often confused, like veridicality and factuality, and introducing the reader to the phenomena in Spanish linguistics on which this thesis focuses.\\

First Section \ref{sect:verandfact} clarifies the difference between veridicality and factuality, provides specific work on each of them, and explains the two main approaches used to study these concepts. Secondly, in Section \ref{sect:spamood}, the first experimental condition in this thesis will be explained, mood in Spanish, together with the specific mood contexts set as experimental conditions. After this, in Section \ref{sect:spec} the second experimental condition, specificity, is introduced, together with the particular specificity contexts whose veridicality is here analyzed.\\

\section{Veridicality and Factuality}
\label{sect:verandfact}

Let us consider examples (\ref{ex:verfac1a}) to (\ref{ex:verfac1d}), where we have events that are intrinsically related. If we were to do an NLI study with these examples, we could directly study the \textit{thruthfulness} of each of them as a single event, i.e., we could study the \textbf{factual nature} of each example towards the real world or the events in the discourse \citep{sauri2009factbank}. Another option would be to study the factual nature of the event \textit{Anna's father has arrived} in the different contexts in which is presented: standing completely on its own (\ref{ex:verfac1a}), or as part of a complex event (\ref{ex:verfac1b}) to (\ref{ex:verfac1d}). In this case, the goal would be not to understand the factuality of \textit{Anna's father has arrived}, but rather to understand how its factuality changes when the event is embedded under an epistemic verb (\ref{ex:verfac1b}), a verb of believe (\ref{ex:verfac1c}), and a verb of speech (\ref{ex:verfac1d}). The former case is a factuality study and the latter a \textbf{veridicality study}, as in the experiments presented here.\\

\begin{exe}
  \ex
    \begin{xlist}
          \item Anna's father has arrived. \label{ex:verfac1a}
          \item John knows that Anna's father has arrived. \label{ex:verfac1b}
          \item John believes that Anna's father has arrived. \label{ex:verfac1c}
          \item John says that Anna's father has arrived. \label{ex:verfac1d}
        \end{xlist}
\end{exe}

Probably one of the most important studies in NLI with a factuality focus is that of \citet{williams2017broad}. They extended the work of \citet{bowman2015large}, which presented an NLI corpus where the premises were crowdsourced figure captions and hypotheses were also crowdsourced, by increasing the number of genres to a total of 10: transcribed conversations, official documents, letters, the public report of 9/11, non-fiction works, popular cultural articles, telephone transcriptions, travel guides, short posts about linguistics, and fiction works. This corpus, denominated as MNLI or Multi-NLI, uses the labels \textit{\{entailment, unknown, contradiction\}} and is now an important benchmark in NLI, as proven, for example, by its use to test BERT \citep{devlin2018bert}. Here, as in this corpus, annotations for premise-hypothesis pairs are crowdsourced. Furthermore, from this corpus originates the XNLI corpus \citep{conneau2018xnli}, which consists of translations of the MLNI corpus into different languages, including Spanish. As we will see in Chapter \ref{chap:main}, a small subset of this corpus was used in the main study.\\

Another important work in NLI is the FactBank corpus \citep{sauri2009factbank}, which consists of $9,488$ manually annotated events by experts which resulted in the FactBank corpus. This means that instead of annotating premise-hypothesis pairs as in the standard NLI task, they annotate events within the sentences they occur. It can be said that this work is done from a factuality focus since it is not designed following different contexts, but the authors indeed present an extensive analysis of the different factors that affect the factuality of an event, that is, they present a  veridicality analysis. Another important feature of their work is their set of labels, which is the baseline of the one used here, and which results from the combination of an epistemic scale, \textit{\{certain, likely (probable), possible\}}, with a quality scale, \textit{\{positive, negative\}}. They map this combination to the traditional Square of Opposition, but for the sake of simplicity, here this square is reduced to a linear scale of factuality, as we will see later on. In any case, the labels they used are the following: \textit{certainly yes} (CT+), \textit{probably yes} (PR+), \textit{possibly yes} (PS+), \textit{certainly not} (CT-), \textit{probably not} (PR-), \textit{possibly not} (PS-), \textit{unknown or uncommitted} (Uu), and \textit{certain but unknown output} (CTu).\\

An interesting factuality study is that of \citet{pavlick2019inherent}. Their goal was to determine whether the disagreement often seen in NLI datasets is noise or a reproducible signal, and thus it should be included in data analysis and modeling. To fulfill this goal, they collected factuality judgments on 500 pairs from different corpora with 50 annotators per pair. But after filtering the annotations, 496 pairs with a mean of 39 workers per pair were left to analyze.\\

When analyzing the annotations, the authors assumed that if the disagreement is noise, the gathered labels can be modeled as a simple Gaussian distribution where the mean is the true label and, consequently, there is only one true label. To verify this assumption, the authors fixed two models for each pair: one single Gaussian and a Gaussian Mixture Model (GMM) where the number of components is chosen during training. Results showed that overall there was a better fit with GMMs and that for $20\%$ of the pairs, there was a nontrivial second component, that is, that for $20\%$ of the pairs, there is not just one \textit{true} label, but rather two. This phenomenon which can be referred to as \textbf{label split} \citep{de2012did} and be caused by \textbf{inherent disagreement}, might help explain the results obtained in the main study.\\

Furthermore, \citet{pavlick2019inherent} analyzed whether context reduces disagreement by collecting annotations at three levels: word, sentence, and paragraph. Results showed that disagreement increases with context. They explain this by hypothesizing that less context may result in higher agreement because with less context humans can more easily call upon \textit{default} interpretations. This supports the decision made here of not including out-of-sentence context in either of the experiments presented here.\\

A clear example of a study on veridicality is that of \citet{ross2019well}. Their goal was to learn whether neural models with no explicit knowledge of verbs' lexical categories can make inferences about veridicality consistent with human inferences. To do so, they crowdsourced human judgments on 1,500 sentences with 137 verb complement constructions using as labels a 5-point Likert scale, and compared these judgments with those made by BERT. Their results and analysis are quite thorough, and led them to conclude that although BERT was able to replicate many of the human judgments, there is still significant room for improvement. Important from this work is also their explanation on the different perspectives that veridicality and factuality studies can have: lexical semantics or sentence meaning approach, and pragmatic or speaker meaning approach. But before explaining their definition of these approaches, we should go over the work of \citet{de2012did}.\\

The study of \citet{de2012did} is a main reference to this thesis. Their goal was to identify some of the linguistic and contextual factors that shape reader's veridicality judgments. To do so, they present crowdsourced annotations on a part of the FactBank corpus with the same set of labels minus one, CTu, and a system for veridicality assesment. For our purposes, the two most important parts of their work is the consideration of the possible occurrence of label split amd the comparison they made between their annotations and the original FactBank's annotations.\\

In order to see if there is a possibility of label split, they studied the plotted distribution of \textbf{agreement patterns}, that is, of how votes for the different labels are distributed in each pair. After discarding the likely noisy patterns, they observed that they are many examples for which is unlikely that the agreement pattern is due to noise. For example, for the premise "In a statement, the White House said it would do “whatever is necessary” to ensure compliance with the sanctions." which had as a hypothesis "there will be compliance with the sanctions", votes were equally split between the labels Uu and PR+, that is, it had an agreement pattern of $[5-5]$. The authors explain this difference by stating that the judgment in this case depends heavily on the speaker's previous knowledge about the White House.\\ 

The comparison they make between the two set of annotations is very important because it shows quantitative differences between what they called, a lexical approach and a pragmatic approach. They do not define in detail the \textbf{lexical approach}, but they do mention that in such approach the context in which we study the factuality of a proposition is a lexical item. \citet{ross2019well} extends this definition by adding that a system following it should aim to model the aspects of a sentence semantics, thus, this representation can be derived from the lexicon and is independent of context, and also, as stated in \citet{sauri2009factbank}, such a system would no consider world knowledge. Thus, linguistic experts are usually the ones who annotate the data.\\ 

As to the \textbf{pragmatic approach}, approach used by \citet{de2012did}, by \citet{ross2019well}, and as hinted in the previous chapter, it is also used here. This approach requires the system to derive a representation of the sentence that considers the communication intent for that sentence in a specific context, that is, a goal-directed  representation of a sentence within the context it was created \citep{ross2019well}. Such a representation entails two important things: the consideration of world knowledge, and the embracing of uncertainty \citep{de2012did}. To obtain this representation a key step is to collect annotations from linguistically naive workers, so for examples (\ref{ex:verfac1a}) to (\ref{ex:verfac1d}) they would consider any knowledge they might have about John, Anna, and her father; and they would ignore any notions as to what \textit{to say} is \textit{supposed} to mean, and instead just use their linguistic intuition.\\

There are other two important facts to consider about these two approaches. The first one is that not everyone explicitly defines the approach they use, but it is often, if not always, possible to infer it. Considering this is very important, specially when comparing results from different studies. The second fact is how these approaches are referred to. \citet{de2012did} and \citet{ross2019well} use the terms lexical and pragmatic, but \citet{ross2019well} also employs the terms sentence meaning approach and speaker meaning approach respectively. Furthermore, \citet{de2012did} hints at the idea that the pragmatic approach is done from the reader's persective, since we want to analyze what the reader understands, not what the author says, as in the lexical approach \citep{sauri2009factbank}. Some studies and tasks like FACT at the \text{Iberian Languages Evaluation Forum} (IberLEF) \citep{rosa2019overview}, make use of these terms, author's or reader's perspective, rather than semantic or pragmatic approach.\\

Aside from the already mentioned Spanish section of the XNLI corpus, there are two other corpora that should be mentioned here: SenSem \citep{fernandez2014sensem} and TAGFACT \citep{fernandez2020tagfact}. Both follow \citet{sauri2009factbank} in annotating events within the sentence they occur instead of annotating premise-hypothesis pairs and having experts instead of lay-workers as here doing the annotation. The difference between the two corpora is that TAGFACT focuses on factuality annotations, whereas SenSem codifies information regarding aspectuality, modality, polarity and factuality.\\

Now that we have seen the differences between a factuality and veridicality focus, and between a lexical and a pragmatic approach, we can go on to exploring another important topic in this thesis: mood in Spanish.\\

\section{Mood in Spanish}
\label{sect:spamood}

Simply put, \textbf{mood} is the grammaticalization of modality \citep{lyons1995linguistic,sanchez2011aproximacion}, and thus it has been traditionally related with the speaker's attitude towards an utterance \citep{lyons1995linguistic,espanola2010nueva}. Although this notion is imprecise \citep{espanola2010nueva} and more needs to be said in order to explain all the phenomena, for our purposes is enough.\\

Since the commitment of the speaker usually takes form in different degrees \citep{lyons1995linguistic}, in most languages mood takes form in different subcategories, like the indicative, the subjunctive and the imperative in Italian, or the subjunctive and the conditional in Hungarian. For Spanish, as in Italian, nowadays most of the grammarians agree on the existence of three subcategories of mood\footnote{For a diachronic review of the study of mood in Spanish see \citet{calvo1995modo}}: indicative, subjunctive and imperative. Since the imperative mood is out of the scope of this thesis, I would simply say that it is the mood mainly used to express commands. As to the indicative and subjunctive mood, the topic is not so clear.\\

Quite often, the indicative and subjunctive moods are defined in opposition to each other \citet{lyons1995linguistic}, and Spanish is not an exception. Some pairs of concepts that \citet{espanola2010nueva} uses to describe this opposition are the following: certainty/uncertainty, reality/virtuality, actuality/non actuality and commitment of the speaker with the veracity of what is spoken/lack of commitment. So for example, in (\ref{ex:indfact}), where the verb \textit{estar} (to be) is in indicative, the reading that we get is that \textit{it is a reality that Sofía was there to see it}. Contrary to this, in (\ref{ex:sbjvnonfact}), where the same verb, \textit{estar}, is in its past perfect tense from the subjunctive, the reading is that \textit{Sofía wasn't there, but we wish there was world, a possible world, in which she was there}. But, as stated in \citet{espanola2010nueva} and \citet{sanchez2011aproximacion}, these oppositions do not always work well.\\

\begin{exe}
  \ex
      \begin{xlist}
          \item {\gll Sofía \textbf{estuvo} allí para verlo. \\ Sofía \textbf{be.\Pst.\Pfv.\Ind.\Tsg{}} there to see.\Inf{}.it\\ \glt "Sofía was there to see it."}\label{ex:indfact}
            \item {\gll ¡Si Sofía \textbf{hubiera estado} allí para verlo!\\ if Sofía \textbf{have.\Pst.\Pfv.\Sbjv.\Tsg{}  be.\Ptcp{}} there to see.\Inf{}.it\\ \glt "If only Sofía had been there to see it!"}\label{ex:sbjvnonfact}
      \end{xlist}
  \ex
    \begin{xlist}
      \item  {\gll Quiero suponer que \textbf{has preparado} todo.\\   want.\Prs.\Ind.\Fsg{} assume.\Inf{} that \textbf{have.\Prs.\Ind.\Ssg{} prepare.\Ptcp{}} everything \\ \glt "I want to assume that you \textbf{have prepared} everything."}\label{ex:indnonfact}
      \item {\gll Siento mucho que se te \textbf{haya averiado} el coche.\\  feel.\Prs.\Ind.\Fsg{} a.lot that itself you.\Dat{} \textbf{have.\Prs.\Sbj.\Tsg{} break.down.\Ptcp{}} the.\M.\Sg{} car \\ \glt "I'm so sorry to hear that your car \textbf{broke down}."}\label{ex:sbjvfact}
    \end{xlist}
\end{exe}

For example, in (\ref{ex:indnonfact}), where the embedded verb \textit{preparar} (to prepare) is in indicative, the reading for the event is \textit{I want the fact that you have prepared everything to be a reality, but it might not be}; and, opposite to this, in (\ref{ex:sbjvfact}), where the embedded verb \textit{averiarse} (to break down) is in the subjunctive mood, its reading is that \textit{it is a reality that your car broke down and I feel sorry for that}. This does not mean that the above mention oppositions are useless, but it does imply that interpreting them too strictly would be a mistake, and that some complimentary considerations are necessary. In this regard, \citet{villalta2008mood} talks about an ordering relation between contextual alternative propositions as the cause for embedded propositions to be in the subjunctive mood, and \citet{mejias1998pragmatic} explains the contrast in terms of old and new information.\\ 

Specifically, \citet{mejias1998pragmatic} understood \textbf{old information} as the information that is pragmatically presupposed, or in other words, the information with which the speaker assumes familiarity. \textbf{New information} would be then the one that is not presupposed. Based on these definitions, he classifies \textbf{matrix verbs}, that is, verbs that take a verbal predicate as a complement, into two groups: those that introduce old information and those that introduce new information. Then he uses this classification to explain the distribution in mood in complements of different types of matrices. The indicative, he claims, is used when the matrix introduces new information, as in the case of mental matrices like \textit{notar} (to notice); and the subjunctive is instead used when the matrice introduces old information, as in the case of comment matrices like  \textit{lamentar} (to regret). Furthermore, he uses this classification to explain the distribution of mood for the following phenomena: the negation of some matrix verbs and matrix verbs whose subject contain a quantifying expression such as \textit{poco/a/s} (a bit/bits) or the adverb \textit{solo} (only). Next, we will briefly go over the syntactic functioning of the indicative and subjunctive moods.\\

As stated in \citet{espanola2010nueva}, authors often talk about a \textbf{dependent} and an \textbf{independent mood}, the first one being the one that requires a grammatical inductor to appear, like in example (\ref{ex:sbjvnonfact}), where the conditional conjunction \textit{si} forces the subjunctive in the verb \textit{estar}, and the second being the one that does not need any grammatical element to appear in the sentence. This distinction mostly correlates with the subjunctive and the indicative moods, since the cases in which the subjunctive appears without depending on any inductor are highly restricted \citep{espanola2010nueva}, and the preferred choice for most of the simple sentences is the indicative, even if in many induced contexts choosing the indicative mood is a requirement, as in example (\ref{ex:indnonfact}).\\

An important characteristic of these induced contexts is that the specific mood, let it be the indicative or the subjunctive, prompted by the grammatical inductor can be an imposition, like in examples (\ref{ex:sbjvnonfact}) and (\ref{ex:sbjvfact}), or a choice, as in examples (\ref{ex:moodaltind}) and (\ref{ex:moodaltsbjv}), and this is what is called \textbf{mood alternation}. This last case is the main feature of the experimental analysis in this thesis, and as such, it will be explained in detail next.\\ 

\subsection{Mood Alternation}
\label{subsect:moodalt}

As stated above, mood alternation is the phenomenon in which a particular mood is induced within a distinct structure, usually the subjunctive in nominal complements, but this induced mood is \textit{optional}, that is, speakers use the structure with either the indicative or the subjunctive mood, as in the examples below which were firstly introduced in Chapter \ref{chap:intr}. As was already explained, there we can see that the two sentences are almost identical, even the translations into English are the same, the only visible difference lies in the morphology of the embedded verb. \textit{Tener} (to have) is in the indicative mood in (\ref{ex:moodaltind}), and in the subjunctive mood in (\ref{ex:moodaltsbjv}). What causes this phenomenon? How is it interpreted by speakers?\\

\begin{exe}
  \ex
    \begin{xlist}
      \item  {\gll El presidente no dijo que el país \textbf{tenía} problemas económicos.\\ the.\M.\Sg{} president.\M.\Sg{} not say.\Pst.\Pfv.\Ind.\Tsg{} that the.\M.\Sg{} country.\M.\Sg{} \textbf{have.\Pst.\Ipfv.\Ind.\Tsg{}} problem.\M.\Pl{} economic.\M.\Pl{} \\\ "The president didn't say that the country \textbf{had} economic problems."\glt }\label{ex:moodaltind}
      \item {\gll El presidente no dijo que el país \textbf{tuviera} problemas económicos.\\   the.\M.\Sg{} president.\M.\Sg{} not say.\Pst.\Pfv.\Ind.\Tsg{} that the.\M.\Sg{} country.\M.\Sg{} \textbf{have.\Pst.\Ipfv.\Sbjv.\Tsg{}} problem.\M.\Pl{} economic.\M.\Pl{} \\\ "The president didn't say that the country had economic problems."\glt }\label{ex:moodaltsbjv}
    \end{xlist}
\end{exe}

Contrary to what we have seen in the previous section about the overall distinction between indicative and subjunctive, in mood alternation the difference is quite clear. As stated in \citet{mejias1998pragmatic,espanola2010nueva} and \citet{falk2017towards}, the indicative is used when the speaker chooses to present the event as new information, and the subjunctive mood is used when the information is already part of the common ground. Thus, in example (\ref{ex:moodaltind}), the fact that \textit{the country has economic problems} is presented as something new to the speaker; and in (\ref{ex:moodaltsbjv}), the fact that \textit{the country has economic problems} or that \textit{the country doesn't have economic problems} is considered to be known by the speaker.\\

An important work on mood alternation is the study of \citet{faulkner2021systematic} from where, as it has already been shown through the references, some theoretical and practical ideas about mood alternation and its study were drawn. Her goals were to see if there are non-standard cases of mood alternation in verbal complements and if this acceptability is dialectal, to understand if the informativeness of the complements affects this acceptability, and if so, to see in which cases this happens. To fulfill these goals she collected acceptability judgments on 128 sentences, with and without context, of speakers from different Spanish varieties. For all the sentences the induced mood is the subjunctive and the alternative is the indicative.\\

The main idea from the results of this study is that mood alternation in verbal complements is a complex phenomenon in which different factors come into play. First of all, the type of complement, or in other words, the type of matrix verb, considerably affects the acceptability judgment. For example, desideratives don't accept the indicative in their complements, but negated epistemics do. Secondly, mood alternation seems to depend on the presence or absence of context and on whether it is informative or not. Thirdly, she shows some dialectal variations which we will neither discuss nor forget here. Lastly, it should be noted that, despite the thorough analysis she presents, not all results could be explained, which shows that further research into mood alternation is needed. Next, the specific mood alternation phenomena analyzed in this thesis are presented.\\

The first type is the negation of the matrix verb, shown in (\ref{ex:moodaltind}) and (\ref{ex:moodaltsbjv}). As stated in \citet{espanola2010nueva}, negation can induce the subjunctive in the verb of the nominal complement. Specifically, they say that the adverb \textit{no} (no) induces the subjunctive mood in the complements of verbs of speech, perception, thought, and belief; and they give examples of mood alternation for verbs of perception. Given the correlation of indicative and subjunctive with new and old information, it is hypothesized that for sentence (\ref{ex:moodaltind}) the speaker considers the embedded event \textit{el país tenía problemas económicos} more certain or likely to be true than in (\ref{ex:moodaltsbjv}). Though, as it will be seen in Section \ref{sect:pildisc}, this effect depends upon the veridicality of the matrix verb. That is, whether the matrix yields the embedded predicate an entailment, a contradiction, or a neutral predicate.\\

\begin{exe}
  \ex
    \begin{xlist}
      \item{\gll Quizás \textbf{dijo la verdad}. \\ maybe \textbf{tell.\Pst.\Pfv.\Ind.\Tsg{}} the.\F.\Sg{} truth.\F.\Sg{} \\ "Maybe s/he \textbf{told} the truth."}\label{ex:moodaltindadv1}
      \item{\gll Quizás \textbf{dijera} la verdad.\\ maybe \textbf{tell.\Pst.\Pfv.\Sbjv.\Tsg{}} the.\F.\Sg{} truth.\F.\Sg{} \\ "Maybe s/he \textbf{told} the truth."}\label{ex:moodaltsbjvadv1}
    \end{xlist}
\end{exe}

The second type of mood alternation phenomena is that of adverbs of doubt and possibility, exemplified in (\ref{ex:moodaltindadv1}) to (\ref{ex:moodaltsbjvadv2}). As stated in \citet{espanola2010nueva}, adverbs of doubt and possibility such as \text{quizás} (maybe) or \textit{probablemente} (probably), induce indicative or subjunctive within their sentences, but the subjunctive mood appears only when the adverb precedes the verb and there is no pause between them, as in examples (\ref{ex:moodaltindadv1}) and (\ref{ex:moodaltsbjvadv1}). To my knowledge, research on this topic is not abundant, thus I was not able to define what exactly that \textit{pause} means, that is, if it is literally a pause like the one indicated by \textit{,} and other elements can be in between, such as the subject as in examples (\ref{ex:moodaltindadv2}) and (\ref{ex:moodaltsbjvadv2}); or if it means that the adverb must inmediately precede the verb and that no element can be in between, as in examples (\ref{ex:moodaltindadv1}) and (\ref{ex:moodaltsbjvadv1}). For the pilot study, the former option was assumed to be true, but knowing that it could \textit{easily} be denied, is the only experiment where this phenomenon is analyzed. As to the veridicality effect, the expectations are similar to the negation case, with the only exception being that here there is no embedded event, and instead were talking about the main event, \textit{(el presidente) dijo la verdad} (the president told the truth). Next, the second feature analyzed in this thesis is presented.\\

\begin{exe}
  \ex
    \begin{xlist}
      \item{\gll Quizás el presidente \textbf{dijo} la verdad.\\ maybe the.\M.\Sg{} president.\M.\Sg{} \textbf{tell.\Pst.\Pfv.\Ind.\Tsg{}} the.\F.\Sg{} truth.\F.\Sg{} \\ "Maybe the president \textbf{told} the truth."}\label{ex:moodaltindadv2}
      \item{\gll Quizás el presidente \textbf{dijera} la verdad.\\ maybe the.\M.\Sg{} president.\M.\Sg{} \textbf{tell.\Pst.\Pfv.\Sbjv.\Tsg{}} the.\F.\Sg{} truth.\F.\Sg{} \\ "Maybe the president \textbf{told} the truth."}\label{ex:moodaltsbjvadv2}
    \end{xlist}
\end{exe}

\section{Specificity}
\label{sect:spec}

If we consider the dialog below we can see a very simple interaction in which \textit{A} informs \textit{B} about the location of a specific book. But that is not everything that is being communicated. If we look at the utterance that \textit{A} produces, we can see that aside from the location of an entity, \textit{A} is referring to two entities: a specific book and a specific table. But what does referring to something mean?\\  

\begin{exe}
  \ex\label{ex:dialog}
  \begin{xlistA}
    \item \textit{The book is on the table.}
    \item \textit{Ok. Thank you.}
  \end{xlistA}
\end{exe}

To \textbf{refer} to something is to point, pick up, or call upon an entity or set of entities \citep{lyons1995linguistic} in the mind of the speaker \citep{caudet1999expresiones}. In other words, when a speaker refers, she presupposes the existence of an entity or set of entities and connects a linguistic expression to it \citep{garcia1998presuposiciones}. When the listener hears such an expression, he tries to make that same connection by corroborating its existence in the physical context, linguistic context, and/or his memory. This is why, the study of referential expressions is intrinsically connected to existence \citep{lyons1995linguistic}, to the study of existential presuppositions; to the point that, as stated in \citet{garcia1998presuposiciones}, the study of existential presuppositions must be supported by the study of reference. Further clarifications about this relation can be seen in \citet{lyons1995linguistic,garcia1998presuposiciones} and \citet{herrasti2011caracteristicas}.\\

Above we have mentioned that the book and table to which \textit{A} refers are a specific book and a specific table. At first glance, this can be simply understood as a particular book and a particular table, but as \citet{caudet1999expresiones} thoroughly explains, the property of being specific has been applied to different concepts, which can lead to some confusion. Thus she puts together the different views on specificity and summarizes them into the following six criteria, criteria that can be used to classify referential expressions:\\

\begin{enumerate}
  \item Existence of the referent in the extra-linguistic reality.
  \item Identifiability of the referent in the extra-linguistic reality.
  \item Extension indicated by the referential expression.
  \item Existence of the referent in the discourse universe.
  \item Identifiability of the referent in the discourse universe.
  \item The expression points to a set of referents pragmatically delimited or chooses a subset of them.
\end{enumerate}

To illustrate the first criterion a good example would be the distinction between "The money I have." and "The money I will have when I'm rich". In the first case "the money" points to real money that the speaker has and that can be counted. Contrary to this, in the second case "the money" is not one that the speaker actually has or that we can feel and count, but rather one that at most exists only in the mind of the speaker. Although this distinction of existing or not in the extra-linguistic reality was shown with a material example, it should be noted that applies to all kinds of entities.\\ 

As to the second criterion, it shows the distinction between pointing to a known and an unknown entity, but it is a pragmatic distinction, it does not affect the existence of the entity. So "my daughter's boyfriend" is still unknown by the speaker in "I will meet my daughter's boyfriend for the first time" and known in "My daughter's boyfriend is again coming for dinner today", but in both cases, he exists.\\

The criterion of extension deals with quantity. Specifically with the number of entities or the amount of matter the referential expression points to. This criterion is exemplified in sentences like "He doesn't have any kids" vs. "He has six kids" or "Most of the flour is on the floor" vs "Almost nothing of the flour is on the floor".\\

The fourth criterion is the one intrinsically connected to coreference and anaphora, so that instead of being built with morphological or syntactical features, it is constructed throughout the discourse and it distinguishes between the entities that have already been introduced and those which are just being presented like "the eldest daughter" in "My daughters have just arrived. The eldest one is crying" vs. "My eldest daughter is crying".\\

Regarding the identifiability of the referent in the discourse universe, it involves the accessibility of the referent, which is shown, for example, in the amount and type of information used in the referral expression. So that when referring to Olaf Scholz, the current German chancellor, we could use the expression "the German chancellor", "the chancellor", or directly "Olaf Scholz".\\  
% Assuming the reference is successful in both cases, in the first case the speaker uses more information because she assumes that in the mind speaker, there is more than one chancellor with equal prominence, and thus more information is needed to ensure the right one is chosen. In the second case, no additional information is needed because the speaker assumes only Olaf Scholz is prominent in the mind of the speaker. Another feature that shows
The last criterion applies to the distinction between "The students will go on the school trip" and "The students who have passed all of their courses will go on the school trip". If this utterance was to be uttered by, for example, a $6^{th}$ grade teacher, in the first case he would be referring to all of his students, whereas in the second one, he would be pointing to only of a subset of his students.\\ 

From these six criteria, here we consider \textbf{specificity} in the sense of the fifth criterion, therefore when manipulating the specificity of a nominal phrase, we are influencing the accessibility of the referent in the discourse universe is what is being analyzed. In particular, here we manipulate it by modifying the amount and type of information given in the subject of the premise.\\ 

Regarding the type of information, two manipulations are performed: individual vs. collective nouns, and common vs. proper nouns. With the first one, we are manipulating the number of individual entities to which we are referring. As it was mentioned in Chapter \ref{chap:intr}, by using an individual noun (in singular) like \textit{president}, we point to one single entity, whereas by using a collective noun like \textit{government}, we point instead to a set of entities; thus when trying to identify the referent in the discourse universe, the set of possible referents the listener considers is different.\\

As to the second manipulation, common vs. proper nouns, we alternate who identifies the referent. When using a common noun like \textit{president}, the speaker only identifies the set of possible entities to which our referent belongs, therefore leaving to the listener the final determination of the particular referent. Contrary to this, when using a proper noun, the speaker \textit{usually}\footnote{For some exceptions on the common use of proper nouns see \citet{caudet1999expresiones}.} points directly to an entity or set of entities. Therefore, by distinguishing between common and proper nouns we have two different ways of identifying the referent: identification by the listener and identification by the speaker.\\

Concerning the amount of information, only one manipulation is considered: is the common noun in the subject modified only by a determiner as in \textit{the president}, or are there any other complements that modify the noun as in \textit{the president of the government}? By performing this manipulation, which here we denominate mixed, we are influencing the retrieval of the referent in a similar way as to the common vs. proper noun distinction since we are modifying the number of possible referents the listener considers. When we have a mixed noun phrase, this number is smaller than with an almost bare noun phrase, but bigger than with a proper noun, thus being a sort of middle ground between common and proper nouns and resulting in the following scale of number of possible referents: common $>$ mixed $>$ proper.\\ 

Regarding the veridicality of these specificity manipulations, it is difficult to define an exact hypothesis since, to my knowledge, there is no work in the matter and thus expectations are surrounded with more uncertainty than mood alternation. Overall it is hypothesized that referring to a set of entities instead of a single one will increase certainty since collective nouns often refer to institutions and organizations for which there is a certain degree of trust, whereas for individual agents subjectivity is assumed to play a greater role and therefore increase uncertainty to even a negative point. Therefore, with a subject like "The police arrested a suspect" a more positive factuality judgment about the fact that the suspect was arrested than in "The police officer arrested a suspect". A similar assumption is applied to the common $>$ mixed $>$ proper scale, with common nouns having a more positive veridicality effect than proper nouns, again due to the subjectivity associated with them.\\